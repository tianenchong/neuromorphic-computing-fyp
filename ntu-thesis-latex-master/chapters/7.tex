\section{Conclusion and Future Work}
Both the neuromorphic network and spintronics have been gaining traction in the field of artificial intelligence and power-efficient, high-performance memory device. The combination of these two is evolutionary in future development and advancement in the field. This project, therefore, explores the practical side of implementation and migration of off-chip trained data to a neuromorphic circuit, and compare the performance between the two, and highlight the challenges and the shortcomings. Small scale neuromorphic network has been performing phenomenally with MNIST dataset at 81.02\% accuracy, sometimes even better than off-chip Tensorflow performance at 80.24\%, but large scale performance suffers due to non-ideality of an opamp. Therefore, immediate future work includes sourcing a higher precision opamp or developing an opamp capable of sustaining its virtual ground voltage, by being immune to both supply voltage and varying input voltage. Further work can include completing the simulation of large scale neuromorphic network once the opamp shortcoming is sorted, full domain wall synapse software Cadence simulation model implementation and simulation, and finally exploring the horizon of domain wall spiked based neuromorphic network, encouraged by benefits such as ultra-low power consumption, as pointed out by the reviewed literature.